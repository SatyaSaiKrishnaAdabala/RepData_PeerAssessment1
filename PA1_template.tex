% Options for packages loaded elsewhere
\PassOptionsToPackage{unicode}{hyperref}
\PassOptionsToPackage{hyphens}{url}
%
\documentclass[
]{article}
\usepackage{lmodern}
\usepackage{amssymb,amsmath}
\usepackage{ifxetex,ifluatex}
\ifnum 0\ifxetex 1\fi\ifluatex 1\fi=0 % if pdftex
  \usepackage[T1]{fontenc}
  \usepackage[utf8]{inputenc}
  \usepackage{textcomp} % provide euro and other symbols
\else % if luatex or xetex
  \usepackage{unicode-math}
  \defaultfontfeatures{Scale=MatchLowercase}
  \defaultfontfeatures[\rmfamily]{Ligatures=TeX,Scale=1}
\fi
% Use upquote if available, for straight quotes in verbatim environments
\IfFileExists{upquote.sty}{\usepackage{upquote}}{}
\IfFileExists{microtype.sty}{% use microtype if available
  \usepackage[]{microtype}
  \UseMicrotypeSet[protrusion]{basicmath} % disable protrusion for tt fonts
}{}
\makeatletter
\@ifundefined{KOMAClassName}{% if non-KOMA class
  \IfFileExists{parskip.sty}{%
    \usepackage{parskip}
  }{% else
    \setlength{\parindent}{0pt}
    \setlength{\parskip}{6pt plus 2pt minus 1pt}}
}{% if KOMA class
  \KOMAoptions{parskip=half}}
\makeatother
\usepackage{xcolor}
\IfFileExists{xurl.sty}{\usepackage{xurl}}{} % add URL line breaks if available
\IfFileExists{bookmark.sty}{\usepackage{bookmark}}{\usepackage{hyperref}}
\hypersetup{
  pdftitle={PA1\_template.Rmd},
  pdfauthor={Satya Sai Krishna Adabala},
  hidelinks,
  pdfcreator={LaTeX via pandoc}}
\urlstyle{same} % disable monospaced font for URLs
\usepackage[margin=1in]{geometry}
\usepackage{color}
\usepackage{fancyvrb}
\newcommand{\VerbBar}{|}
\newcommand{\VERB}{\Verb[commandchars=\\\{\}]}
\DefineVerbatimEnvironment{Highlighting}{Verbatim}{commandchars=\\\{\}}
% Add ',fontsize=\small' for more characters per line
\usepackage{framed}
\definecolor{shadecolor}{RGB}{248,248,248}
\newenvironment{Shaded}{\begin{snugshade}}{\end{snugshade}}
\newcommand{\AlertTok}[1]{\textcolor[rgb]{0.94,0.16,0.16}{#1}}
\newcommand{\AnnotationTok}[1]{\textcolor[rgb]{0.56,0.35,0.01}{\textbf{\textit{#1}}}}
\newcommand{\AttributeTok}[1]{\textcolor[rgb]{0.77,0.63,0.00}{#1}}
\newcommand{\BaseNTok}[1]{\textcolor[rgb]{0.00,0.00,0.81}{#1}}
\newcommand{\BuiltInTok}[1]{#1}
\newcommand{\CharTok}[1]{\textcolor[rgb]{0.31,0.60,0.02}{#1}}
\newcommand{\CommentTok}[1]{\textcolor[rgb]{0.56,0.35,0.01}{\textit{#1}}}
\newcommand{\CommentVarTok}[1]{\textcolor[rgb]{0.56,0.35,0.01}{\textbf{\textit{#1}}}}
\newcommand{\ConstantTok}[1]{\textcolor[rgb]{0.00,0.00,0.00}{#1}}
\newcommand{\ControlFlowTok}[1]{\textcolor[rgb]{0.13,0.29,0.53}{\textbf{#1}}}
\newcommand{\DataTypeTok}[1]{\textcolor[rgb]{0.13,0.29,0.53}{#1}}
\newcommand{\DecValTok}[1]{\textcolor[rgb]{0.00,0.00,0.81}{#1}}
\newcommand{\DocumentationTok}[1]{\textcolor[rgb]{0.56,0.35,0.01}{\textbf{\textit{#1}}}}
\newcommand{\ErrorTok}[1]{\textcolor[rgb]{0.64,0.00,0.00}{\textbf{#1}}}
\newcommand{\ExtensionTok}[1]{#1}
\newcommand{\FloatTok}[1]{\textcolor[rgb]{0.00,0.00,0.81}{#1}}
\newcommand{\FunctionTok}[1]{\textcolor[rgb]{0.00,0.00,0.00}{#1}}
\newcommand{\ImportTok}[1]{#1}
\newcommand{\InformationTok}[1]{\textcolor[rgb]{0.56,0.35,0.01}{\textbf{\textit{#1}}}}
\newcommand{\KeywordTok}[1]{\textcolor[rgb]{0.13,0.29,0.53}{\textbf{#1}}}
\newcommand{\NormalTok}[1]{#1}
\newcommand{\OperatorTok}[1]{\textcolor[rgb]{0.81,0.36,0.00}{\textbf{#1}}}
\newcommand{\OtherTok}[1]{\textcolor[rgb]{0.56,0.35,0.01}{#1}}
\newcommand{\PreprocessorTok}[1]{\textcolor[rgb]{0.56,0.35,0.01}{\textit{#1}}}
\newcommand{\RegionMarkerTok}[1]{#1}
\newcommand{\SpecialCharTok}[1]{\textcolor[rgb]{0.00,0.00,0.00}{#1}}
\newcommand{\SpecialStringTok}[1]{\textcolor[rgb]{0.31,0.60,0.02}{#1}}
\newcommand{\StringTok}[1]{\textcolor[rgb]{0.31,0.60,0.02}{#1}}
\newcommand{\VariableTok}[1]{\textcolor[rgb]{0.00,0.00,0.00}{#1}}
\newcommand{\VerbatimStringTok}[1]{\textcolor[rgb]{0.31,0.60,0.02}{#1}}
\newcommand{\WarningTok}[1]{\textcolor[rgb]{0.56,0.35,0.01}{\textbf{\textit{#1}}}}
\usepackage{graphicx,grffile}
\makeatletter
\def\maxwidth{\ifdim\Gin@nat@width>\linewidth\linewidth\else\Gin@nat@width\fi}
\def\maxheight{\ifdim\Gin@nat@height>\textheight\textheight\else\Gin@nat@height\fi}
\makeatother
% Scale images if necessary, so that they will not overflow the page
% margins by default, and it is still possible to overwrite the defaults
% using explicit options in \includegraphics[width, height, ...]{}
\setkeys{Gin}{width=\maxwidth,height=\maxheight,keepaspectratio}
% Set default figure placement to htbp
\makeatletter
\def\fps@figure{htbp}
\makeatother
\setlength{\emergencystretch}{3em} % prevent overfull lines
\providecommand{\tightlist}{%
  \setlength{\itemsep}{0pt}\setlength{\parskip}{0pt}}
\setcounter{secnumdepth}{-\maxdimen} % remove section numbering

\title{PA1\_template.Rmd}
\author{Satya Sai Krishna Adabala}
\date{8/8/2020}

\begin{document}
\maketitle

\hypertarget{r-markdown}{%
\subsection{R Markdown}\label{r-markdown}}

This is an R Markdown document. Markdown is a simple formatting syntax
for authoring HTML, PDF, and MS Word documents. For more details on
using R Markdown see \url{http://rmarkdown.rstudio.com}.

When you click the \textbf{Knit} button a document will be generated
that includes both content as well as the output of any embedded R code
chunks within the document. You can embed an R code chunk like this:

\begin{Shaded}
\begin{Highlighting}[]
\NormalTok{knitr}\OperatorTok{::}\NormalTok{opts_chunk}\OperatorTok{$}\KeywordTok{set}\NormalTok{(}\DataTypeTok{echo =} \OtherTok{TRUE}\NormalTok{)}
\CommentTok{# code for filtering NA`s and ploting the histogram}
\NormalTok{t}\OperatorTok{$}\NormalTok{date <-}\StringTok{ }\KeywordTok{as.Date}\NormalTok{(t}\OperatorTok{$}\NormalTok{date)}
\CommentTok{#class(t$date)}
\NormalTok{t2 <-}\StringTok{ }\KeywordTok{subset}\NormalTok{(t,t}\OperatorTok{$}\NormalTok{steps}\OperatorTok{!=}\StringTok{ 'NA'}\NormalTok{)}
\CommentTok{#summary(t2)}
\NormalTok{y <-}\StringTok{ }\KeywordTok{unique}\NormalTok{(t2}\OperatorTok{$}\NormalTok{date)}
\NormalTok{x<-}\StringTok{ }\KeywordTok{data.frame}\NormalTok{(}\DataTypeTok{date =}\NormalTok{y[}\DecValTok{1}\NormalTok{], }\DataTypeTok{steps =}\DecValTok{0}\NormalTok{)}
\CommentTok{#sum(t2$steps[t2$date ==y[1]])}

\ControlFlowTok{for}\NormalTok{ (i }\ControlFlowTok{in} \DecValTok{1}\OperatorTok{:}\StringTok{ }\KeywordTok{length}\NormalTok{(y))}
\NormalTok{\{}
  
\NormalTok{  x1 <-}\StringTok{ }\KeywordTok{data.frame}\NormalTok{(date <-}\StringTok{ }\KeywordTok{unique}\NormalTok{(t2}\OperatorTok{$}\NormalTok{date[t2}\OperatorTok{$}\NormalTok{date }\OperatorTok{==}\StringTok{ }\NormalTok{y[i]]),}
\NormalTok{  steps <-}\StringTok{ }\KeywordTok{sum}\NormalTok{(t2}\OperatorTok{$}\NormalTok{steps[t2}\OperatorTok{$}\NormalTok{date }\OperatorTok{==}\StringTok{ }\NormalTok{y[i]]))}
  \KeywordTok{names}\NormalTok{(x1) <-}\StringTok{ }\KeywordTok{c}\NormalTok{(}\StringTok{"date"}\NormalTok{,}\StringTok{"steps"}\NormalTok{)}
\NormalTok{  x<-}\StringTok{ }\KeywordTok{rbind}\NormalTok{(x,x1)}
  
\NormalTok{\}}
\NormalTok{x <-}\StringTok{ }\NormalTok{x[}\DecValTok{2}\OperatorTok{:}\KeywordTok{nrow}\NormalTok{(x),]}
\KeywordTok{plot}\NormalTok{(x}\OperatorTok{$}\NormalTok{date,x}\OperatorTok{$}\NormalTok{steps,}\DataTypeTok{main=}\StringTok{"test"}\NormalTok{,}\DataTypeTok{xlab=}\StringTok{"Dates"}\NormalTok{,}\DataTypeTok{ylab=}\StringTok{"frequency"}\NormalTok{,}\DataTypeTok{type =} \StringTok{"h"}\NormalTok{,}\DataTypeTok{lwd=}\DecValTok{4}\NormalTok{)}
\end{Highlighting}
\end{Shaded}

\includegraphics{PA1_template_files/figure-latex/code-1.pdf}

\begin{Shaded}
\begin{Highlighting}[]
\NormalTok{knitr}\OperatorTok{::}\NormalTok{opts_chunk}\OperatorTok{$}\KeywordTok{set}\NormalTok{(}\DataTypeTok{echo =} \OtherTok{TRUE}\NormalTok{)}
\KeywordTok{mean}\NormalTok{(x}\OperatorTok{$}\NormalTok{steps)}
\end{Highlighting}
\end{Shaded}

\begin{verbatim}
## [1] 10766.19
\end{verbatim}

\begin{Shaded}
\begin{Highlighting}[]
\KeywordTok{median}\NormalTok{(x}\OperatorTok{$}\NormalTok{steps)}
\end{Highlighting}
\end{Shaded}

\begin{verbatim}
## [1] 10765
\end{verbatim}

\begin{Shaded}
\begin{Highlighting}[]
\NormalTok{z <-}\StringTok{ }\KeywordTok{unique}\NormalTok{(t2}\OperatorTok{$}\NormalTok{interval)}
\NormalTok{y2 <-}\StringTok{ }\KeywordTok{data.frame}\NormalTok{(}\DataTypeTok{interval=}\DecValTok{0}\NormalTok{,}\DataTypeTok{average_steps=}\DecValTok{0}\NormalTok{)}
\ControlFlowTok{for}\NormalTok{ (i }\ControlFlowTok{in} \DecValTok{1}\OperatorTok{:}\StringTok{ }\KeywordTok{length}\NormalTok{(z))}
\NormalTok{\{}
  
\NormalTok{  x1 <-}\StringTok{ }\KeywordTok{data.frame}\NormalTok{(interval <-}\StringTok{ }\KeywordTok{unique}\NormalTok{(t2}\OperatorTok{$}\NormalTok{interval[t2}\OperatorTok{$}\NormalTok{interval }\OperatorTok{==}\StringTok{ }\NormalTok{z[i]]),}
\NormalTok{                   average_steps <-}\StringTok{ }\KeywordTok{mean}\NormalTok{(t2}\OperatorTok{$}\NormalTok{steps[t2}\OperatorTok{$}\NormalTok{interval }\OperatorTok{==}\StringTok{ }\NormalTok{z[i]]))}
  
  \KeywordTok{names}\NormalTok{(x1) <-}\StringTok{ }\KeywordTok{c}\NormalTok{(}\StringTok{"interval"}\NormalTok{,}\StringTok{"average_steps"}\NormalTok{)}
\NormalTok{  y2<-}\StringTok{ }\KeywordTok{rbind}\NormalTok{(y2,x1)}
  
\NormalTok{\}}

\NormalTok{y2 <-}\StringTok{ }\NormalTok{y2[}\DecValTok{2}\OperatorTok{:}\KeywordTok{nrow}\NormalTok{(y2),]}

\KeywordTok{plot}\NormalTok{(y2}\OperatorTok{$}\NormalTok{interval,y2}\OperatorTok{$}\NormalTok{average_steps,}\DataTypeTok{type=}\StringTok{"l"}\NormalTok{,}\DataTypeTok{xlab=}\StringTok{"Time Seriesinterval of 5 minutes"}\NormalTok{, }\DataTypeTok{ylab=} \StringTok{"Average Steps"}\NormalTok{)}
\end{Highlighting}
\end{Shaded}

\includegraphics{PA1_template_files/figure-latex/code to find mean and median for a day and average steps per interval-1.pdf}

\begin{Shaded}
\begin{Highlighting}[]
\NormalTok{knitr}\OperatorTok{::}\NormalTok{opts_chunk}\OperatorTok{$}\KeywordTok{set}\NormalTok{(}\DataTypeTok{echo =} \OtherTok{TRUE}\NormalTok{)}
\NormalTok{c <-}\StringTok{ }\KeywordTok{which}\NormalTok{(}\KeywordTok{is.na}\NormalTok{(t}\OperatorTok{$}\NormalTok{steps))}
\KeywordTok{length}\NormalTok{(c)}
\end{Highlighting}
\end{Shaded}

\begin{verbatim}
## [1] 2304
\end{verbatim}

\begin{Shaded}
\begin{Highlighting}[]
\NormalTok{t}\OperatorTok{$}\NormalTok{steps[}\KeywordTok{is.na}\NormalTok{(t}\OperatorTok{$}\NormalTok{steps) }\OperatorTok{==}\StringTok{ }\OtherTok{TRUE}\NormalTok{] <-}\StringTok{ }\NormalTok{y2}\OperatorTok{$}\NormalTok{average_steps[y2}\OperatorTok{$}\NormalTok{interval }\OperatorTok{==}\StringTok{ }\DecValTok{5}\NormalTok{]}
\end{Highlighting}
\end{Shaded}

\begin{Shaded}
\begin{Highlighting}[]
\NormalTok{knitr}\OperatorTok{::}\NormalTok{opts_chunk}\OperatorTok{$}\KeywordTok{set}\NormalTok{(}\DataTypeTok{echo =} \OtherTok{TRUE}\NormalTok{)}
\CommentTok{# including NA`s }
\CommentTok{## Histogram to show number of steps per day`}
\NormalTok{y <-}\StringTok{ }\KeywordTok{unique}\NormalTok{(t}\OperatorTok{$}\NormalTok{date)}
\NormalTok{x<-}\StringTok{ }\KeywordTok{data.frame}\NormalTok{(}\DataTypeTok{date =}\NormalTok{y[}\DecValTok{1}\NormalTok{], }\DataTypeTok{steps =}\DecValTok{0}\NormalTok{)}
\CommentTok{#sum(t$steps[t$date ==y[1]])}

\ControlFlowTok{for}\NormalTok{ (i }\ControlFlowTok{in} \DecValTok{1}\OperatorTok{:}\StringTok{ }\KeywordTok{length}\NormalTok{(y))}
\NormalTok{\{}
  
\NormalTok{  x1 <-}\StringTok{ }\KeywordTok{data.frame}\NormalTok{(date <-}\StringTok{ }\KeywordTok{unique}\NormalTok{(t}\OperatorTok{$}\NormalTok{date[t}\OperatorTok{$}\NormalTok{date }\OperatorTok{==}\StringTok{ }\NormalTok{y[i]]),}
\NormalTok{                   steps <-}\StringTok{ }\KeywordTok{sum}\NormalTok{(t}\OperatorTok{$}\NormalTok{steps[t}\OperatorTok{$}\NormalTok{date }\OperatorTok{==}\StringTok{ }\NormalTok{y[i]]))}
  \KeywordTok{names}\NormalTok{(x1) <-}\StringTok{ }\KeywordTok{c}\NormalTok{(}\StringTok{"date"}\NormalTok{,}\StringTok{"steps"}\NormalTok{)}
\NormalTok{  x<-}\StringTok{ }\KeywordTok{rbind}\NormalTok{(x,x1)}
  
\NormalTok{\}}
\CommentTok{#x}
\CommentTok{#names(x1)}
\NormalTok{x <-}\StringTok{ }\NormalTok{x[}\DecValTok{2}\OperatorTok{:}\KeywordTok{nrow}\NormalTok{(x),]}
\NormalTok{x}\OperatorTok{$}\NormalTok{steps <-}\StringTok{ }\KeywordTok{ceiling}\NormalTok{(x}\OperatorTok{$}\NormalTok{steps)}
\CommentTok{#x$steps}
\KeywordTok{plot}\NormalTok{(x}\OperatorTok{$}\NormalTok{date,x}\OperatorTok{$}\NormalTok{steps,}\DataTypeTok{main=}\StringTok{"test"}\NormalTok{,}\DataTypeTok{xlab=}\StringTok{"Dates"}\NormalTok{,}\DataTypeTok{ylab=}\StringTok{"frequency"}\NormalTok{,}\DataTypeTok{type =} \StringTok{"h"}\NormalTok{,}\DataTypeTok{lwd=}\DecValTok{4}\NormalTok{)}
\end{Highlighting}
\end{Shaded}

\includegraphics{PA1_template_files/figure-latex/code to plot histogram after imputing the NA`s for number of steps per day and to calculate meand and median-1.pdf}

\begin{Shaded}
\begin{Highlighting}[]
\KeywordTok{mean}\NormalTok{(x}\OperatorTok{$}\NormalTok{steps)}
\end{Highlighting}
\end{Shaded}

\begin{verbatim}
## [1] 9367.082
\end{verbatim}

\begin{Shaded}
\begin{Highlighting}[]
\KeywordTok{median}\NormalTok{(x}\OperatorTok{$}\NormalTok{steps)}
\end{Highlighting}
\end{Shaded}

\begin{verbatim}
## [1] 10395
\end{verbatim}

\begin{Shaded}
\begin{Highlighting}[]
\NormalTok{knitr}\OperatorTok{::}\NormalTok{opts_chunk}\OperatorTok{$}\KeywordTok{set}\NormalTok{(}\DataTypeTok{echo =} \OtherTok{TRUE}\NormalTok{)}
\NormalTok{t3 <-}\StringTok{ }\KeywordTok{subset}\NormalTok{(t,}\KeywordTok{weekdays}\NormalTok{(t}\OperatorTok{$}\NormalTok{date) }\OperatorTok{==}\StringTok{ }\KeywordTok{c}\NormalTok{(}\StringTok{"Sunday"}\NormalTok{,}\StringTok{"Saturday"}\NormalTok{))}
\NormalTok{t4 <-}\StringTok{ }\KeywordTok{subset}\NormalTok{(t,}\KeywordTok{weekdays}\NormalTok{(t}\OperatorTok{$}\NormalTok{date) }\OperatorTok{!=}\StringTok{ }\KeywordTok{c}\NormalTok{(}\StringTok{"Sunday"}\NormalTok{,}\StringTok{"Saturday"}\NormalTok{))}

\CommentTok{# Weekdays Intervals}

\NormalTok{z <-}\StringTok{ }\KeywordTok{unique}\NormalTok{(t4}\OperatorTok{$}\NormalTok{interval)}
\NormalTok{y2 <-}\StringTok{ }\KeywordTok{data.frame}\NormalTok{(}\DataTypeTok{interval=}\DecValTok{0}\NormalTok{,}\DataTypeTok{average_steps=}\DecValTok{0}\NormalTok{)}
\ControlFlowTok{for}\NormalTok{ (i }\ControlFlowTok{in} \DecValTok{1}\OperatorTok{:}\StringTok{ }\KeywordTok{length}\NormalTok{(z))}
\NormalTok{\{}
  
\NormalTok{  x1 <-}\StringTok{ }\KeywordTok{data.frame}\NormalTok{(interval <-}\StringTok{ }\KeywordTok{unique}\NormalTok{(t4}\OperatorTok{$}\NormalTok{interval[t4}\OperatorTok{$}\NormalTok{interval }\OperatorTok{==}\StringTok{ }\NormalTok{z[i]]),}
\NormalTok{                   average_steps <-}\StringTok{ }\KeywordTok{mean}\NormalTok{(t4}\OperatorTok{$}\NormalTok{steps[t4}\OperatorTok{$}\NormalTok{interval }\OperatorTok{==}\StringTok{ }\NormalTok{z[i]]))}
  
  \KeywordTok{names}\NormalTok{(x1) <-}\StringTok{ }\KeywordTok{c}\NormalTok{(}\StringTok{"interval"}\NormalTok{,}\StringTok{"average_steps"}\NormalTok{)}
\NormalTok{  y2<-}\StringTok{ }\KeywordTok{rbind}\NormalTok{(y2,x1)}
  
\NormalTok{\}}

\NormalTok{y2 <-}\StringTok{ }\NormalTok{y2[}\DecValTok{2}\OperatorTok{:}\KeywordTok{nrow}\NormalTok{(y2),]}

\NormalTok{z <-}\StringTok{ }\KeywordTok{unique}\NormalTok{(t3}\OperatorTok{$}\NormalTok{interval)}
\NormalTok{y4 <-}\StringTok{ }\KeywordTok{data.frame}\NormalTok{(}\DataTypeTok{interval=}\DecValTok{0}\NormalTok{,}\DataTypeTok{average_steps=}\DecValTok{0}\NormalTok{)}
\ControlFlowTok{for}\NormalTok{ (i }\ControlFlowTok{in} \DecValTok{1}\OperatorTok{:}\StringTok{ }\KeywordTok{length}\NormalTok{(z))}
\NormalTok{\{}
  
\NormalTok{  x1 <-}\StringTok{ }\KeywordTok{data.frame}\NormalTok{(interval <-}\StringTok{ }\KeywordTok{unique}\NormalTok{(t3}\OperatorTok{$}\NormalTok{interval[t3}\OperatorTok{$}\NormalTok{interval }\OperatorTok{==}\StringTok{ }\NormalTok{z[i]]),}
\NormalTok{                   average_steps <-}\StringTok{ }\KeywordTok{mean}\NormalTok{(t3}\OperatorTok{$}\NormalTok{steps[t3}\OperatorTok{$}\NormalTok{interval }\OperatorTok{==}\StringTok{ }\NormalTok{z[i]]))}
  
  \KeywordTok{names}\NormalTok{(x1) <-}\StringTok{ }\KeywordTok{c}\NormalTok{(}\StringTok{"interval"}\NormalTok{,}\StringTok{"average_steps"}\NormalTok{)}
\NormalTok{  y4<-}\StringTok{ }\KeywordTok{rbind}\NormalTok{(y4,x1)}
  
\NormalTok{\}}

\NormalTok{y4 <-}\StringTok{ }\NormalTok{y4[}\DecValTok{2}\OperatorTok{:}\KeywordTok{nrow}\NormalTok{(y4),]}

\KeywordTok{par}\NormalTok{(}\DataTypeTok{mfrow=}\KeywordTok{c}\NormalTok{(}\DecValTok{2}\NormalTok{,}\DecValTok{1}\NormalTok{),}\DataTypeTok{mar =}\KeywordTok{c}\NormalTok{(}\DecValTok{4}\NormalTok{,}\DecValTok{4}\NormalTok{,}\DecValTok{2}\NormalTok{,}\DecValTok{1}\NormalTok{))}

\KeywordTok{plot}\NormalTok{(y4}\OperatorTok{$}\NormalTok{interval,y4}\OperatorTok{$}\NormalTok{average_steps,}\DataTypeTok{type=}\StringTok{"l"}\NormalTok{,}\DataTypeTok{xlab=}\StringTok{"Time Series interval of 5 minutes"}\NormalTok{, }\DataTypeTok{ylab=} \StringTok{"Average Steps Weekends"}\NormalTok{,}\DataTypeTok{main=}\StringTok{"Weekends"}\NormalTok{)}

\KeywordTok{plot}\NormalTok{(y2}\OperatorTok{$}\NormalTok{interval,y2}\OperatorTok{$}\NormalTok{average_steps,}\DataTypeTok{type=}\StringTok{"l"}\NormalTok{,}\DataTypeTok{xlab=}\StringTok{"Time Series interval of 5 minutes"}\NormalTok{, }\DataTypeTok{ylab=} \StringTok{"Average Steps Weekdays"}\NormalTok{,}\DataTypeTok{main =}\StringTok{"Weekdays"}\NormalTok{)}
\end{Highlighting}
\end{Shaded}

\includegraphics{PA1_template_files/figure-latex/code for calculating the time series with average steps for weekdays and weekends-1.pdf}

\end{document}
